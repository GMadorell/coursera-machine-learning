
\RequirePackage[l2tabu, orthodox]{nag} % Complaints if syntax isn't right.
\documentclass[10pt]{extarticle}

\usepackage[utf8]{inputenc}
\usepackage{extsizes} % Allow for more sizes such as 14pt or 17pt on document class.
\usepackage{mathtools} % Allows conditional math expressions, etc.
\usepackage{amsfonts} % Allows \mathbb{R}, which is the real numbers symbol.
\usepackage[export]{adjustbox} % Wraps \includegraphics with more keys (options).





\usepackage[top=3.5cm, bottom=3.5cm, left=3.5cm, right=3.5cm]{geometry}

\usepackage{microtype} % Makes document more readable by optimizing space between letters.

\usepackage{units} % Adds \nicefrac{1}{2}: (½), \unit[5]{m}{s}: 5 m/s.

\usepackage[colorlinks=false, pdfborder={0 0 0}]{hyperref} % Allow clicking references and the table of contents in pdfs.
\usepackage{cleveref} % Adds fig/formula etc. before references (use \cref).

%%% New commands.

% Don't output references in case they're empty - http://tex.stackexchange.com/questions/74476/how-to-avoid-empty-thebibliography-environment-bibtex-if-there-are-no-refere

\let\myBib\thebibliography
\let\endmyBib\endthebibliography

\renewcommand\thebibliography[1]{\ifx\relax#1\relax\else\myBib{#1}\fi}


% Norm -> ||something||
\newcommand{\norm}[1]{\left\lVert#1\right\rVert}

% Stdfig -> Used as \stdfig{width}{label_name}{caption}
% Requires: image called 'caption' in img folder.
% Output: A figure with the given width, labeled as 'fig:label_name'

\newcommand{\stdfig}[3]{
    \begin{figure}
    \centering
    \includegraphics[width = #1]{img/#2.eps}
    \caption{#3}
    \label{fig:#2}
    \end{figure}
}



\begin{document} 



%% Front page.
\title{10.02 - Application Example: Photo OCR}

    
    \date{}
    

    

    \maketitle

\newpage
%% Abstract page.


%% Table of contents page.



%% Body start.
\section{Problem Description and
Pipeline}\label{problem-description-and-pipeline}

Photo OCR = Photo Optical Character Recognizion. This problem focuses on
how to get computers to read text in pictures that we take.

\subsection{Pipeline}\label{pipeline}

\begin{enumerate}
\def\labelenumi{\arabic{enumi}.}
\itemsep1pt\parskip0pt\parsep0pt
\item
  Text detection - need to find where text is located in the photo.
\item
  Character segmentation - divide the photo rectangle into different
  letters.
\item
  Character classification - see which letter is which.
\end{enumerate}

We can see the Machine Learning pipeline in
\cref{fig:photo_ocr_pipeline}. One of the principal steps when building
a ML system, it's very important to split the model into modules so
different people can work on different things.

\stdfig{12cm}{photo_ocr_pipeline}{Photo OCR pipeline}

\section{Sliding Windows}\label{sliding-windows}

\subsection{Text Detection}\label{text-detection}

First step of the pipeline is \emph{text detection}. \smallskip

We can first explain how to build a pedestrian detection system (easier)
and then incorporate the ideas into the text detection module.

What we do is take some standarized (same size) images where there are
pedestrians and some where there are not
(\cref{fig:pedestrian_detection_training_data}) and train with that.
What we can do is then take that data and classify it as a supervised
learning problem.

When we are predicting from an image, we simply take a rectangle and
shift it around the image, testing each time whether there is a
pedestrian there or not, as seen in
\cref{fig:pedestrian_sliding_window}. We start with a small window size,
move it around the image and then try larger image patches (we do this
by actually making the img smaller, not the window larger). \smallskip

\stdfig{12cm}{pedestrian_detection_training_data}{Data used for pedestrian detection}

\stdfig{10cm}{pedestrian_sliding_window}{Example of how to apply a sliding window when recognizing pedestrians}

When doing text detection, we do a similar thing. We also use positive
examples and negative examples (\cref{fig:text_detection_training_data})
to build up a supervised learning environment.

We then apply the classifier using the sliding window technique and
collect the results. We then use computer vision techniques: use
expansion on the collected results image. Afterwards, we take advantage
of knowing that text boxes should have width much higher than height and
discard those that don't satisfy this characteristic. We can then just
place bounding boxes around the rectangles that are left, and finally
cut those image regions. We can see an example of this process in
\cref{fig:text_detection_visual_algorithm}.

\stdfig{12cm}{text_detection_training_data}{Data used for text detection}

\stdfig{12cm}{text_detection_visual_algorithm}{Text detection steps}

\subsection{Character Segmentation}\label{character-segmentation}

For separating the individual characters inside an image, we can use a
similar technique than when we were doing text detection.

We can just take positive and negative examples and train a classifier
with that, as seen in \cref{fig:character_segmentation_training_data}.
The only difference is that now we try to find whether there is a
separation between two characters as the true label. After training the
classifier, we use a 1D sliding window (only from left to right) and
collect the results (which will be the splits), as we can see in
\cref{fig:character_segmentation_sliding_window}.

\stdfig{12cm}{character_segmentation_training_data}{Data used for training a character classifier}

\stdfig{8cm}{character_segmentation_sliding_window}{Results of applying 1D sliding window for character segmentation}

\subsection{Character Classification}\label{character-classification}

The last step of the pipeline is just to use the previously done
classification task so we know which letter corresponds to each of the
split symbols.

\section{Getting Lots of Data and Artificial
Data}\label{getting-lots-of-data-and-artificial-data}

\textbf{Artificial Data Synthesis} = either create data from scratch or
amplify an already existing dataset. \smallskip

For example, on the character recognition task, we can synthesize data
to help us learn better. One way to create data from scratch would be to
simply take existing fonts and throw them to a random background similar
to the ones from real data. We can also rotate or blur them to make them
more realistic. This is a very creative job, as we can see. An example
to amplify the existing dataset is to introduce small distortions on the
existing letter examples (\cref{fig:real_data_distortion}). \smallskip

It usually does not help to add purely random/meaningless noise to the
data. We need to only add noise that is expected on the test set.
\smallskip

If we had a speech recognition task we could add for example crowd on
the background to the clean data examples.

\stdfig{10cm}{real_data_distortion}{Synthesizing data via real data amplification using distortions}

\subsection{Discussion on getting more
data}\label{discussion-on-getting-more-data}

\begin{enumerate}
\def\labelenumi{\arabic{enumi}.}
\item
  Make sure you have a low bias classifier (plot learning curves to know
  that). For example: keep increasing the number of features/number of
  hidden units in neural network until you have a low bias classifier.
  If we don't do this, getting more data isn't likely to help.
\item
  Ask yourself: ``How much work would it be to get 10x as much data as
  we currently have?'' If the answer is reasonable, consider doing that.
\item
  Thing about crowd sourcing the data labeling process. Buy hours from
  people, basically. Example: Amazon Mechanical Turk.
\end{enumerate}

\section{Ceiling Analysis: What Part of the Pipeline to Work on
Next}\label{ceiling-analysis-what-part-of-the-pipeline-to-work-on-next}

When developing on a ML system, one of the most valuable resources is
the time of the developers. We need to know what to prioritize when
working on a pipeline. \smallskip

We will continue taking a look on the OCR pipeline
(\cref{fig:photo_ocr_pipeline}). Where to alocate resources (time)? The
first step is to get a number to evaluate the whole system (accuracy,
recall, etc.). We can then evaluate every single module using the same
score, using as test data manually selected ground truth. This is
basically giving the correct answers to the algorithm, as if the other
parts of the pipeline had a 100\% score. Doing this consecutively we
might end up with the results seen in \cref{fig:ocr_initial_results}. In
this example, we see that the part that needs more work is text
detection, as the drop is a 17\%, much higher than the other parts.

\stdfig{10cm}{ocr_initial_results}{Photo OCR results}


%% Body end.

%% Bibliography.

\clearpage



    \nocite{*}

\bibliographystyle{plain}
\bibliography{references}

\end{document}
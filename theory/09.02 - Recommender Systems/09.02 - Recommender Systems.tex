
\RequirePackage[l2tabu, orthodox]{nag} % Complaints if syntax isn't right.
\documentclass[10pt]{extarticle}

\usepackage[utf8]{inputenc}
\usepackage{extsizes} % Allow for more sizes such as 14pt or 17pt on document class.
\usepackage{mathtools} % Allows conditional math expressions, etc.
\usepackage{amsfonts} % Allows \mathbb{R}, which is the real numbers symbol.
\usepackage[export]{adjustbox} % Wraps \includegraphics with more keys (options).


% For writing pseudocode snippets: http://ctan.mackichan.com/macros/latex/contrib/algorithmicx/algorithmicx.pdf
\usepackage{algorithm}
\usepackage{algpseudocode}




\usepackage[top=3.5cm, bottom=3.5cm, left=3.5cm, right=3.5cm]{geometry}

\usepackage{microtype} % Makes document more readable by optimizing space between letters.

\usepackage{units} % Adds \nicefrac{1}{2}: (½), \unit[5]{m}{s}: 5 m/s.

\usepackage[colorlinks=false, pdfborder={0 0 0}]{hyperref} % Allow clicking references and the table of contents in pdfs.
\usepackage{cleveref} % Adds fig/formula etc. before references (use \cref).

%%% New commands.

% Don't output references in case they're empty - http://tex.stackexchange.com/questions/74476/how-to-avoid-empty-thebibliography-environment-bibtex-if-there-are-no-refere

\let\myBib\thebibliography
\let\endmyBib\endthebibliography

\renewcommand\thebibliography[1]{\ifx\relax#1\relax\else\myBib{#1}\fi}


% Norm -> ||something||
\newcommand{\norm}[1]{\left\lVert#1\right\rVert}

% Stdfig -> Used as \stdfig{width}{label_name}{caption}
% Requires: image called 'caption' in img folder.
% Output: A figure with the given width, labeled as 'fig:label_name'

\newcommand{\stdfig}[3]{
    \begin{figure}
    \centering
    \includegraphics[width = #1]{img/#2.eps}
    \caption{#3}
    \label{fig:#2}
    \end{figure}
}



\begin{document} 



%% Front page.
\title{09.02 - Recommender Systems}

    
    \date{}
    

    

    \maketitle

\newpage
%% Abstract page.


%% Table of contents page.



%% Body start.
\section{Problem Formulation}\label{problem-formulation}

Example: Predicting movie ratings.

Imagine a company where users rate movies using zero to five stars.

Notation:

\begin{itemize}
\itemsep1pt\parskip0pt\parsep0pt
\item
  $n_u$ = number of users
\item
  $n_m$ = number of movies
\item
  $r(i, j)$ = 1 if user $j$ has rated movie $i$
\item
  $y^{(i,j)}$ = rating given by user $j$ to movie $i$ (defined only if
  $r(i, j)$ = 1)
\end{itemize}

The problem is: given this dataset, to look through all the missing
ratings on the dataset and fill them (\cref{fig:movie_ratings_table}).

\stdfig{9cm}{movie_ratings_table}{Movie ratings table}

\section{Content Based
Recommendations}\label{content-based-recommendations}

How do we predict the ``?'' in the movies table
(\cref{fig:movie_ratings_table})?

Let's guess that we have two additional features in the table: one that
measures the amount of romance a movie has and another one that measures
the amount of action (\cref{fig:movie_ratings_table_with_features}).

\stdfig{11cm}{movie_ratings_table_with_features}{Movie ratings table with features}

If we then add the interceptor term $x_0 = 1$, we have a feature vector
for each movie ($x^{(1)} = [1, \, 0.9, \,  0]^T$, for example). \bigskip

The idea is to use linear regression for each different user and predict
the missing parameters with that. More formally: for each user $j$,
learn a parameter $\theta^{(j)} \in \mathbb{R}^3$. Predict user $j$ as
rating movie $i$ with $(\theta^{(j)})^Tx^{(i)}$ stars. \smallskip

For example, in \cref{fig:movie_ratings_table_with_features}, if we
wanted to predict ``Cute pupies of love'' for ``Alice'', and we already
got $\theta$, we would do: $(\theta^{(1)})^Tx^{(3)}$. \smallskip

\subsection{Optimization Objective}\label{optimization-objective}

We can describe this intention (learn $\theta$ for the user $j$) with
the formula:

\[ \min_{\theta^{(j)}}  \frac{1}{2} \sum_{i:r(i, j)=1} ((\theta^{(j)})^Tx^{(i)} - y^{(i,j)})^2 + \frac{\lambda}{2} \sum_{k=1}^n(\theta_k^{(j)})^2 \]

Where $\sum_{i:r(i, j)=1} $ is a for loop with all the rated movies by
that user. \smallskip

To learn all the $\theta$, we can do:

\[ \min_{\theta^{(1)}, \dots, \theta^{(n_u)}}  \frac{1}{2} \sum_{j=1}^{n_u} \sum_{i:r(i, j)=1} ((\theta^{(j)})^Tx^{(i)} - y^{(i,j)})^2 + \frac{\lambda}{2} \sum_{j=1}^{n_u} \sum_{k=1}^n(\theta_k^{(j)})^2 \]

Which is very, very similar to the standard linear regression equation
(only filtering the unknows votes out).

\subsubsection{Gradient descent update}\label{gradient-descent-update}

\[ \theta_k^{(j)} := \theta_k^{(j)} - \alpha \sum_{i:r(i, j)=1} \left( (\theta^{(j)})^Tx^{(i)} - y^{(i,j)}\right) x_k^{(i)} \,\, \text{(for $k$ = 0)} \]

\[ \theta_k^{(j)} := \theta_k^{(j)} - \alpha \left( \sum_{i:r(i, j)=1} \left( (\theta^{(j)})^Tx^{(i)} - y^{(i,j)}\right) x_k^{(i)} + \lambda\theta_k^{(j)} \right) \,\, \text{(for $k \neq$ 0)} \]

\section{Collaborative Filtering}\label{collaborative-filtering}

This algorithm has the characteristic of learning the features it needs
by himself. \smallskip

In the last example (\cref{fig:movie_ratings_table_with_features}) we
had some features. Extracting those features is quite time consuming,
though. \smallskip

In this new algorithm, we won't have the features
(\cref{fig:movie_ratings_table_unknown_features}). Let's imagine that
the users tells us whether they like romance and action movies (the
$\theta$). Then, it's possible to infer the values of $x_1$ and $x_2$.
\smallskip 

\stdfig{12cm}{movie_ratings_table_unknown_features}{Move ratings with unknown features}

We can exploit that we know the ratings of the users and their
preferences to know which value to assign to each movie feature. For
example, if a user says that he likes action movies and he rates a movie
very high, we can assume that that movie will have action.

\subsection{Optimization Algorithm}\label{optimization-algorithm}

Given $\theta^{(1)}, \dots, \theta^{(n_u)}$, to learn $x^{(i)}$:

\[ \min_{x^{(i)}}  \frac{1}{2} \sum_{i:r(i, j)=1} ((\theta^{(j)})^Tx^{(i)} - y^{(i,j)})^2 + \frac{\lambda}{2} \sum_{k=1}^n(x_k^{(i)})^2 \]

Which reads as: choosing the features so that the predicted value will
be similar to the actual value that we observe on the rating of the
users. \smallskip

We can then compute this for all the movies. Given
$\theta^{(1)}, \dots, \theta^{(n_u)}$, to learn
$x^{(1)}, \dots, x^{(n_m)}$:

\[ \min_{x^{(1)}, \dots, x^{(n_m)}}  \frac{1}{2} \sum_{i=1}^{n_m} \sum_{j:r(i, j)=1} ((\theta^{(j)})^Tx^{(i)} - y^{(i,j)})^2 + \frac{\lambda}{2} \sum_{i=1}^{n_m} \sum_{k=1}^n(x_k^{(i)})^2 \]

\subsubsection{Gradient descend update}\label{gradient-descend-update}

\[ x_k^{(i)} := x_k^{(i)} - \alpha \sum_{j:r(i, j)=1} \left( (\theta^{(j)})^Tx^{(i)} - y^{(i,j)}\right) \theta_k^{(j)} \,\, \text{(for $k$ = 0)} \]

\[ x_k^{(i)} := x_k^{(i)} - \alpha \left( \sum_{j:r(i, j)=1} \left( (\theta^{(j)})^Tx^{(i)} - y^{(i,j)}\right) \theta_k^{(j)} + \lambda x_k^{(i)} \right) \,\, \text{(for $k \neq$ 0)} \]

\subsection{How to continue}\label{how-to-continue}

We've seen in content based filtering that given
$x^{(1)}, \dots, x^{(n_m)}$ we can estimate
$\theta^{(1)}, \dots, \theta^{(n_u)}$. \smallskip

In collaborative filtering, given $\theta^{(1)}, \dots, \theta^{(n_u)}$,
we can estimate $x^{(1)}, \dots, x^{(n_m)}$. \bigskip

So, this is the egg-chicken problem. What we can do, therefore, is
randomly guess $\theta$ to learn $x$ (features). We can then use the $x$
to learn $\theta$, etc.

\section{Collaborative Filtering
Algorithm}\label{collaborative-filtering-algorithm}

We've talked about the ideas of how if you're given features for movies,
we can learn parameters $\theta$ for users. We've also seen how if we're
given parameters for the users we can use that to extract features for
the movies. We'll now find an algorithm to solve that problem. \bigskip

We could to the chicken-egg back and forth optimization algorithm, but
there is an algorithm that optimizes both $x$ and $\theta$ at the same
time, so it's better. \smallskip

The cost function is:

\[ J(x^{(1)}, \dots, x^{(n_m)}, \theta^{(1)}, \dots, \theta^{(n_u)}) = \frac{1}{2} \sum_{(i,j):r(i, j)=1} \left( (\theta^{(j)})^Tx^{(i)} - y^{(i,j)}\right)^2 + \frac{\lambda}{2} \sum_{i=1}^{n_m} \sum_{k=1}^n(x_k^{(i)})^2  + \frac{\lambda}{2} \sum_{j=1}^{n_u} \sum_{k=1}^n(\theta_k^{(j)})^2\]

Which is basically merging both cost functions (collaborative and
content based filtering) into one, taking advantage that the less
squares sumation is the same in both and then adding the respective
regularization term.

We also suppress the interceptor term $x_1 = 1$, as the algorithm will
learn it by himself if it's really needed.

\subsection{Gradient descent}\label{gradient-descent}

The same as before for the $\theta$ and $x$ respectively.

\subsection{Detailed algorithm}\label{detailed-algorithm}

We can find a more detailed explanation in
\cref{alg:collaborative_filtering}.

\begin{algorithm}
\caption{Collaborative filtering algorithm} \label{alg:collaborative_filtering}
\begin{algorithmic}[1]
\State Initialize $x^{(1)}, \dots, x^{(n_m)}, \theta^{(1)}, \dots, \theta^{(n_u)}$ to small random values.
\State Minimize $J(x^{(1)}, \dots, x^{(n_m)}, \theta^{(1)}, \dots, \theta^{(n_u)})$ using gradient descent (or an advanced optimization algorithm).

For every $j = 1, \dots, n_u, i = 1, \dots, n_m$:

$ x_k^{(i)} := x_k^{(i)} - \alpha \left( \sum_{j:r(i, j)=1} \left( (\theta^{(j)})^Tx^{(i)} - y^{(i,j)}\right) \theta_k^{(j)} + \lambda x_k^{(i)} \right) $

$ \theta_k^{(j)} := \theta_k^{(j)} - \alpha \left( \sum_{i:r(i, j)=1} \left( (\theta^{(j)})^Tx^{(i)} - y^{(i,j)}\right) x_k^{(i)} + \lambda\theta_k^{(j)} \right) $

\State For a user with parameters $\theta$ and a movie with (learned) features $x$, predict a star rating of $\theta^Tx$.

\end{algorithmic}
\end{algorithm}

\section{Vectorization: Low Rank Matrix
Factorization}\label{vectorization-low-rank-matrix-factorization}

We'll see an alternative algorithm for solving the collaborative
filtering problem. \bigskip

Take the dataset we have (\cref{fig:movie_ratings_table}), and put it
directly onto a matrix called $Y$. Then, we can stack the $x$ vectors
and the $\theta$ vectors in two matrices, $X$ and $\Theta$. Afterwards,
we can compute the prediction of the ratings simply by doing
$X\Theta^T$. All this can be seen in
\cref{fig:low_rank_matrix_factorization}

\stdfig{10cm}{low_rank_matrix_factorization}{Low rank matrix factorization}

\subsection{Finding related movies}\label{finding-related-movies}

For each product $i$, we learn a feature vector
$x^{(i)} \in \mathbb{R}^n$.

How to find 5 movies $j$ most related to movie $i$?

Find the 5 movies $j$ with the smallest $\norm{x^{(i)} - x^{(j)}}$.


%% Body end.

%% Bibliography.

\clearpage



    \nocite{*}

\bibliographystyle{plain}
\bibliography{references}

\end{document}
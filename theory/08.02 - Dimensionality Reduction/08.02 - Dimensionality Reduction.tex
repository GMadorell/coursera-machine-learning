
\RequirePackage[l2tabu, orthodox]{nag} % Complaints if syntax isn't right.
\documentclass[10pt]{extarticle}

\usepackage[utf8]{inputenc}
\usepackage{extsizes} % Allow for more sizes such as 14pt or 17pt on document class.
\usepackage{mathtools} % Allows conditional math expressions, etc.
\usepackage{amsfonts} % Allows \mathbb{R}, which is the real numbers symbol.
\usepackage[export]{adjustbox} % Wrapd \includegraphics with more keys (options).

% For writing pseudocode snippets: http://ctan.mackichan.com/macros/latex/contrib/algorithmicx/algorithmicx.pdf
\usepackage{algorithm}
\usepackage{algpseudocode}

% Layout the document: http://texdoc.net/texmf-dist/doc/latex/geometry/geometry.pdf
\usepackage[top=3.5cm, bottom=3.5cm, left=3.5cm, right=3.5cm]{geometry}

\usepackage{microtype} % Makes document more readable by optimizing space between letters.

\usepackage{units} % Adds \nicefrac{1}{2}: (½), \unit[5]{m}{s}: 5 m/s.

\usepackage[colorlinks=false, pdfborder={0 0 0}]{hyperref} % Allow clicking references and the table of contents in pdfs.
\usepackage{cleveref} % Adds fig/formula etc. before references (use \cref).

%%% New commands.

% Don't output references in case they're empty - http://tex.stackexchange.com/questions/74476/how-to-avoid-empty-thebibliography-environment-bibtex-if-there-are-no-refere

\let\myBib\thebibliography
\let\endmyBib\endthebibliography

\renewcommand\thebibliography[1]{\ifx\relax#1\relax\else\myBib{#1}\fi}


% Norm -> ||something||
\newcommand{\norm}[1]{\left\lVert#1\right\rVert}

% Stdfig -> Used as \stdfig{label_name}{caption}
% Requires: image called 'caption' in img folder.
% Output: A figure with horizontal size the same as the document,
%         labeled as 'fig:label_name'

\newcommand{\stdfig}[2]{
    \begin{figure}
    \centering
    \includegraphics[max size={\textwidth}{\textheight}]{img/#1.eps}
    \caption{#2}
    \label{fig:#1}
    \end{figure}
}


\begin{document} 



%% Front page.
\title{Dimensionality Reduction}

    
    \date{}
    

    

    \maketitle

\newpage
%% Abstract page.


%% Table of contents page.



%% Body start.
\section{Motivation I: Data
Compression}\label{motivation-i-data-compression}

Dimensionality reduction means transforming from a given amount of
features to a smaller amount of them (for example: 2D -\textgreater{}
1D). This is useful when we have redunant features (such as inches and
cm) that we can delete and have the same performance but being faster.

\section{Motivation II: Data
Visualization}\label{motivation-ii-data-visualization}

As we can only visualize data that has three or less dimensions, we can
use dimensionality reduction in order to visualize datasets that have
\textgreater{}3 dimensions. The hard part is how to select which
features represent the data better.

\section{Principal Component Analysis Problem
Formulation}\label{principal-component-analysis-problem-formulation}

\begin{itemize}
\itemsep1pt\parskip0pt\parsep0pt
\item
  Needs feature scaling before being applied.
\item
  PCA tries to minimize the projection error.
\end{itemize}

Problem: Reduce from 2D to 1D: Find a direction onto which to project
the data so as to minimize the projection error. \smallskip

If we have more dimensions than that, we will find n vectors (where n is
the amount of dimensions we want to end with) indicating the directions
into which to project the data, minimizing the projection error.
\smallskip

Basically, PCA is trying to find a lower dimensional surface onto which
to project the data so as to minimize the squared projection error
(squared distance between each point and the location in which it gets
projected).

\subsection{PCA is not lineal
regression}\label{pca-is-not-lineal-regression}

Linear regression fits a line according to the predictions of the
hypothesis (minimizes mse between data and the hyphotesis), whereas PCA
minimizes the squared error directly onto the projection vector.

\section{PCA Algorithm}\label{pca-algorithm}

Remember to preprocess the data normalizing it before applying PCA!
\smallskip

En example of PCA from 2D to 1D can be seen in \cref{fig:pca_2d_to_1d}.
We need to know how to calculate the vector $u^{(1)}$ and all $z^{(i)}$.
\smallskip

\stdfig{pca_2d_to_1d}{Example of PCA from 2D to 1D}

The algorithm for finding out principal components can be found in
\cref{alg:pca}.

\begin{algorithm}
\caption{PCA algorithm}\label{alg:pca}
\begin{algorithmic}[1]
\Function{PCA}{$x, k$} \Comment{x = unlabeled dataset}
    \State Reduces data from $n$-dimensions to $k$-dimensions.
    \State Compute covariance matrix:
        \State \indent $\Sigma = \frac{1}{m} \sum_{(i=1)}^n (x^{(i)})(x^{(i)})^T$
    \State Compute "eigenvectors" of matrix $\Sigma$:
        \State \indent [$U, S, V$] = svd(Sigma);  \Comment{Singular Value Decomposition}
        \State \indent We only need $U$. $k$-first $U$ vectors are the ones that we will use to describe the data.
        \State \indent $U_{\text{reduce}} \gets \text{First} \, k \, \text{vectors from} \, U \, \text{(in columns)}$
    \State $z \gets U_{\text{reduce}}^T x$
    \State \Return $z$
\EndFunction
\end{algorithmic}
\end{algorithm}

\section{Choosing the number of principal
components}\label{choosing-the-number-of-principal-components}

Typically, we choose $k$ to be the smallest value that still retains the
99\% of the variance. That is the same as: \[
\frac{\text{average squared projection error}}{\text{total variation in the data}} = \frac{\frac{1}{m} \sum_{i=1}^m \norm{x^{(i)} - x_\text{approx}^{(i)}}^2}{\frac{1}{m} \sum_{i=1}^m \norm{x^{(i)}}^2} \leq 0.01 \, (1\%)
\]

Then, we can select $k$ iteratively, starting from one and trying every
single value until we retain 99\% of the variance, at which point we can
select the $k$ that satisfies that restriction. \smallskip

Above solution is very slow though. We can take advantage that we are
computing the ``svd''. We can use the $S$ matrix returned by it to
calculate the retained variance in a much more easy way:

\[
\text{For a given} \, k:
\frac{\text{average squared projection error}}{\text{total variation in the data}} = 1 - \frac{\sum_{(i=1)}^k Sii}{\sum_{(i=1)}^n Sii}
\]

Using above formula, we can simply calculate the svd once and iterate
with the $S$ matrix until we find the correct $k$.

\section{Deconstruction from Compressed
Representation}\label{deconstruction-from-compressed-representation}

We're going to see how to reconstruct the data after having applied PCA
to it. It's a lossy method, so we're not going to get the same exact
data, just an approximation of it.

\[
X_\text{approx} = U_\text{reduce} z
\]

\section{Advice for Appylying PCA}\label{advice-for-appylying-pca}

We can use PCA in a supervised machine learning problem (using only $X$)
to speed up the algorithm. Just remember to also use the PCA with the
predict data! \smallskip

The PCA mapping should be defined using only the training set, as we
could bias it running it with the cv/test dataset as well.


%% Body end.

%% Bibliography.

    \nocite{*}

\bibliographystyle{plain}
\bibliography{references}

\end{document}

\RequirePackage[l2tabu, orthodox]{nag} % Complaints if syntax isn't right.
\documentclass[10pt]{extarticle}

\usepackage[utf8]{inputenc}
\usepackage{extsizes} % Allow for more sizes such as 14pt or 17pt on document class.
\usepackage{mathtools} % Allows conditional math expressions, etc.
\usepackage{amsfonts} % Allows \mathbb{R}, which is the real numbers symbol.
\usepackage[export]{adjustbox} % Wraps \includegraphics with more keys (options).


% For writing pseudocode snippets: http://ctan.mackichan.com/macros/latex/contrib/algorithmicx/algorithmicx.pdf
\usepackage{algorithm}
\usepackage{algpseudocode}




\usepackage[top=3.5cm, bottom=3.5cm, left=3.5cm, right=3.5cm]{geometry}

\usepackage{microtype} % Makes document more readable by optimizing space between letters.

\usepackage{units} % Adds \nicefrac{1}{2}: (½), \unit[5]{m}{s}: 5 m/s.

\usepackage[colorlinks=false, pdfborder={0 0 0}]{hyperref} % Allow clicking references and the table of contents in pdfs.
\usepackage{cleveref} % Adds fig/formula etc. before references (use \cref).

%%% New commands.

% Don't output references in case they're empty - http://tex.stackexchange.com/questions/74476/how-to-avoid-empty-thebibliography-environment-bibtex-if-there-are-no-refere

\let\myBib\thebibliography
\let\endmyBib\endthebibliography

\renewcommand\thebibliography[1]{\ifx\relax#1\relax\else\myBib{#1}\fi}


% Norm -> ||something||
\newcommand{\norm}[1]{\left\lVert#1\right\rVert}

% Stdfig -> Used as \stdfig{width}{label_name}{caption}
% Requires: image called 'caption' in img folder.
% Output: A figure with the given width, labeled as 'fig:label_name'

\newcommand{\stdfig}[3]{
    \begin{figure}
    \centering
    \includegraphics[width = #1]{img/#2.eps}
    \caption{#3}
    \label{fig:#2}
    \end{figure}
}



\begin{document} 



%% Front page.
\title{09.01 - Anomaly Detection}

    
    \date{}
    

    

    \maketitle

\newpage
%% Abstract page.


%% Table of contents page.



%% Body start.
\section{Problem Motivation}\label{problem-motivation}

Example:\\Say that we have some aircraft engine features:

\begin{itemize}
\itemsep1pt\parskip0pt\parsep0pt
\item
  $x_1$ = heat generated
\item
  $x_2$ = vibration intensity
\end{itemize}

In this type of problem, we want to know if a new engine is ``wrong'',
meaning that it might not function properly and we need to further test
it.

As we can see in \cref{fig:aircraft_engines}, if the new engine is
inside of the ``normal'' motors, then we assume it's also normal. On the
other hand, if the engine falls in a region that is outside of the
normal one, we consider it as being ``wrong''. The exact problem, then,
is defining what is this normal region and how to decided whether a
given point is inside it or not. \bigskip

\stdfig{8cm}{aircraft_engines}{Anomaly detection in aircraft engines}

More formally, we're given a dataset
$\{x^{(1)}, x^{(2)}, \dots, x^{(m)}\}$ of normal data points and we need
to decide whether $x_{test}$ is anomalous.

We use probability models that estimate $P(x_{test} = \text{normal})$.

\section{Gaussian Distribution}\label{gaussian-distribution}

We say that $x \sim \mathcal{N}(\mu, \sigma^2)$ (x is distributed as
Normal), with mean $\mu$ and variance $\sigma^2$ if:

\[ p(x;\mu,\sigma^2) = \frac{1}{\sqrt{2\pi}\sigma} \exp(- \frac{(x - \mu)^2}{2\sigma^2}) \]

Which can be modelled as a bell curve, as seen in
\cref{fig:normal_distribution}. In simple terms, $\mu$ is where the
center of the curve is and $\sigma^2$ is how wide the curve is.

\stdfig{8cm}{normal_distribution}{Normal distribution}

\section{Algorithm}\label{algorithm}

Training set: $\{x^{(1)}, x^{(2)}, \dots, x^{(m)}\}$.\\Each example is
$x \in \mathbb{R}^n$.

We're going to model the probabilities as such:

\[ p(x) = p(x_1;\mu_1,\sigma_1^2) \, p(x_2;\mu_2,\sigma_2^2) \dots p(x_n;\mu_n,\sigma_n^2) = \prod_{j=1}^n p(x_j; \mu_j, \sigma_j^2) \]

Which is basically to estimate each feature as a normal distribution and
join them together. This formula assumes independent features, but it
works well even though the assumption might be wrong.

Putting it all together, we have \cref{alg:anomaly_detection_normals}.

\begin{algorithm}
\caption{Anomaly detection algorithm using gaussians} \label{alg:anomaly_detection_normals}
\begin{algorithmic}[1]
\State Choose features $x_i$ that might be indicative of anomalous examples.
\State Fit parameters $\mu_1, \dots, \mu_n$, $\sigma_1^2, \dots, \sigma^2_n$.
\State \indent $\mu_j = \frac{1}{m}\sum_{i=1}^m x_j^{(i)}$
\State \indent $\sigma^2_j = \frac{1}{m}\sum_{i=1}^m (x_j^{(i)} - \mu_j)^2$ 
\State Given a new example $x$, compute:
\State \indent $p(x) = \prod_{j=1}^n p(x_j; \mu_j, \sigma_j^2) = \prod_{j=1}^n \frac{1}{\sqrt{2\pi}\sigma} \exp(- \frac{(x_j - \mu_j)^2}{2\sigma_j^2}) $
\State Anomaly if $p(x) < \epsilon$
\end{algorithmic}
\end{algorithm}

\section{Developing and Evaluating and Anomaly Detection
System}\label{developing-and-evaluating-and-anomaly-detection-system}

When developing a learning algorithm making decisions is much easier if
we have a way of evaluating our learning algorithm. \smallskip

To evaluate our anomaly detection system we need some labeled data, of
anomalous and non-anomalous examples.

We still train with only normal/non-anomalous examples. The cv set and
the test set contain anomalous data points as well as non-anomalous.
Then we just fit the model and evaluate it as normal supervised problem
(f1-score, recall, etc.). We can also use cross validation set to choose
the threshold $\epsilon$.

\section{Anomaly Detection vs.~Supervised
Learning}\label{anomaly-detection-vs.supervised-learning}

If we have labeled data, why don't we just use a supervised learning
straight away? One of the reasons is that anomaly detection works better
with very small anomalies (positive examples) whereas supervised
learning works better when we have a large number of positive and
negative examples.

Also, anomaly detection works better when we have many differet types of
anomalies, so that supervised learning doesn't learn from positive
examples what the anomalies look like. Future anomalies might look
nothing like any of the anomalous examples we've seen so far.

\section{Choosing What Features to
Use}\label{choosing-what-features-to-use}

Before running algorithm plot histogram of the data. If the result is a
non gaussian plot, try to transform it (use log(x), log(x+constant),
$x^c$, etc..) and, if after transforming it it looks gausian, substitute
the whole feature for its transformation. \bigskip

A good way to get new features is to look at anamolous examples
missclassified manually to search what differences they have compared to
the normal examples. \bigskip

Another good idea is to choose features that might take on unusually
large or small values in the event of an anormally.

\section{Multivariate Gaussian
Distribution}\label{multivariate-gaussian-distribution}

Problem of the anterior algorithm is that it does not take into account
the other features. It evaluates every feature differently.

This algorithm, on the other hand, doesn't model $p(x_1), p(x_2) \dots $
separetly, it models $p(x)$ in one go. This means that, to model a
feature, it uses inormation from the rest, so it might work better.
Parameters: $\mu \in \mathbb{R}^n, \Sigma \in \mathbb{R}^{n \times n}$
(covariance matrix).

\[ p(x;\mu,\Sigma) = \frac{1}{(2\pi)^{\nicefrac{n}{2}} | \Sigma | ^{\nicefrac{1}{2}}} \exp(-\frac{1}{2} (x-\mu)^T \Sigma^{-1} (x-\mu)) \]

\section{Anomaly Detection using the Multivariate Gaussian
Distribution}\label{anomaly-detection-using-the-multivariate-gaussian-distribution}

Parameter fitting:

\[ \mu = \frac{1}{m} \sum_{i=1}^m x^{(i)} \]
\[ \Sigma = \frac{1}{m} \sum_{i=1}^m (x^{(i)} - \mu)(x^{(i)} - \mu)^T\]

After we have done this, we do the same as the normal anomaly detection
algorithm but we calculate $p(x)$ with the new formula. \bigskip

This new model captures correlations between features automatically.
It's also computationally more expensive. Furthermore, it needs $m > n$
(more samples than features), otherwise $\Sigma$ is non-invertible.


%% Body end.

%% Bibliography.

\clearpage



    \nocite{*}

\bibliographystyle{plain}
\bibliography{references}

\end{document}

\RequirePackage[l2tabu, orthodox]{nag} % Complaints if syntax isn't right.
\documentclass[10pt]{extarticle}

\usepackage[utf8]{inputenc}
\usepackage{extsizes} % Allow for more sizes such as 14pt or 17pt on document class.
\usepackage{mathtools} % Allows conditional math expressions, etc.
\usepackage{amsfonts} % Allows \mathbb{R}, which is the real numbers symbol.
\usepackage{graphicx} % Allows \includegraphics.

% For writing pseudocode snippets: http://ctan.mackichan.com/macros/latex/contrib/algorithmicx/algorithmicx.pdf
\usepackage{algorithm}
\usepackage{algpseudocode}

% Layout the document: http://texdoc.net/texmf-dist/doc/latex/geometry/geometry.pdf
\usepackage[top=3.5cm, bottom=3.5cm, left=3.5cm, right=3.5cm]{geometry}

\usepackage{microtype} % Makes document more readable by optimizing space between letters.

\usepackage{units} % Adds \nicefrac{1}{2}: (½), \unit[5]{m}{s}: 5 m/s.

\usepackage[colorlinks=false, pdfborder={0 0 0}]{hyperref} % Allow clicking references and the table of contents in pdfs.
\usepackage{cleveref} % Adds fig/formula etc. before references (use \cref).

%%% New commands.

% Don't output references in case they're empty - http://tex.stackexchange.com/questions/74476/how-to-avoid-empty-thebibliography-environment-bibtex-if-there-are-no-refere

\let\myBib\thebibliography
\let\endmyBib\endthebibliography

\renewcommand\thebibliography[1]{\ifx\relax#1\relax\else\myBib{#1}\fi}


% Norm -> ||something||
\newcommand{\norm}[1]{\left\lVert#1\right\rVert}

% Stdfig -> Used as \stdfig{label_name}{caption}
% Requires: image called 'caption' in img folder.
% Output: A figure with horizontal size the same as the document,
%         labeled as 'fig:label_name'

\newcommand{\stdfig}[2]{
    \begin{figure}
    \centering
    \includegraphics[width=\textwidth]{img/#1.eps}
    \caption{#2}
    \label{fig:#1}
    \end{figure}
}


\begin{document} 



%% Front page.
\title{Clustering}

    
    \date{}
    

    

    \maketitle

\newpage
%% Abstract page.


%% Table of contents page.



%% Body start.
\section{Unsupervised Learning:
Introduction}\label{unsupervised-learning-introduction}

Clustering will be our first unsupervised algorithm.

In clustering, we will no longer have labels for the data. We will make
the algorithm find some structure in the data, clusters.

Applications:

\begin{itemize}
\itemsep1pt\parskip0pt\parsep0pt
\item
  Market segmentation.
\item
  Social network analysis.
\item
  Organize computing clusters.
\item
  Astronomical data analysis.
\end{itemize}

\section{K-Means Algorithm}\label{k-means-algorithm}

Iterative algorithm for grouping data in clusters:

Input = number of clusters and unlabeled training set.

\begin{enumerate}
\def\labelenumi{\arabic{enumi}.}
\itemsep1pt\parskip0pt\parsep0pt
\item
  Randomly initialize cluster centroids.
\item
  Assign each data point to its closest centroid.
\item
  Move each centroid to the mean of the points assigned to them.
\item
  Iterate 2 and 3 until no major change in step 3 happens.
\end{enumerate}

A more formal description is given in \cref{alg:kmeans}:

\begin{algorithm}
\caption{Kmeans algorithm}\label{alg:kmeans}
\begin{algorithmic}[1]
\Function{Kmeans}{$K, x$} \Comment{K = number of clusters, x = unlabeled dataset}
    \State Randomly initialize $K$ cluster centroids $\mu_1, \mu_2, ... , \mu_K \in \mathbb{R}$.
    \While{not convergence}
        \For{$i$ = 1 to $m$} \Comment{m is the length of the dataset}
            \State $c^{(i)} \gets$ index of cluster centroid closest to $x^{(i)}$
        \EndFor 
        \For{$k$ = 1 to $K$} 
            \State $\mu_k \gets$ average (mean) of points assigned to cluster $k$
        \EndFor
    \EndWhile
\EndFunction
\end{algorithmic}
\end{algorithm}

To find out which cluster is the closest one, we can use any distance
(such as the euclidian one).

Convergence is declared once there is no change between two consecutive
runs.

If a cluster ends with no points assigned to it, simply eliminate it or
reinitialize it randomly if the amount of clusters can't be diminished
at all.

\section{Optimization Objective}\label{optimization-objective}

The algorithms we've seen all have a cost function to optimize. Turns
out, Kmeans also has a cost function. Let's remember some of the kmean
notation first:

\begin{itemize}
\itemsep1pt\parskip0pt\parsep0pt
\item
  $c^{(i)} = \text{index of cluster to which example} x^{(i)} \text{is currently assigned}$
\item
  $\mu_k = \text{cluster centroid } k (\mu_k \in \mathbb{R}^n)$
\item
  $\mu_c(i) = \text{cluster centroid of cluster to which example } x^{(i)} \text{ has been assigned}$
\end{itemize}

Given that, we can now see that kmeans is minimizing
\cref{eq:kmeans_cost_function}

\begin{equation} \label{eq:kmeans_cost_function}
J(c^{(1)}, \dots, c^{(m)}, \mu_1, \dots, \mu_K) = \frac{1}{m} \sum_{i=1}^{m} \norm{x^{(i)} - \mu_c(i)}^2
\end{equation}

\cref{eq:kmeans_cost_function} is sometimes called the distortion
formula.

\section{Random Initialization}\label{random-initialization}

We need to make sure that the kmeans algorithm evades local optimal.
Some things that help with that are:

\begin{itemize}
\itemsep1pt\parskip0pt\parsep0pt
\item
  $K < m$: number of clusters \textless{} number of data points.
\item
  Randomly choosing $K$ training examples instead of choosing freely
  random points for cluster initialization.
\end{itemize}

Even using that, it might end up in a local optima. We can initialize
kmeans several times to solve that as shown in
\cref{alg:kmeans_no_local_optima}.

\begin{algorithm}
\caption{Kmeans solving local optima problem}\label{alg:kmeans_no_local_optima}
\begin{algorithmic}[1]
\Function{Kmeans}{$K, x, repetitions$} \Comment{K = number of clusters, x = unlabeled dataset}
    \For{$i$ = 1 to repetitions}
        \State Run normal K-means.
    \EndFor
    \State \Return the instance that gets the lowest cost function.
\EndFunction
\end{algorithmic}
\end{algorithm}

\section{Choosing the number of
clusters}\label{choosing-the-number-of-clusters}

The most common thing to do is to explore that data in order to know the
optimum amount of clusters. \smallskip

A method sometimes used is called Elbow number
(\cref{fig:elbow_method}), in which you pick the point in which the cost
function no longer decreases very fast. This can only be applied in
certain cases in which that value is very clear, which might not be
always as shown in the right graph of \cref{fig:elbow_method}. \bigskip

\stdfig{elbow_method}{Graphs indicating when to use the elbow method to select the optimal amount of clusters.}

Sometimes Kmeans is used to get clusters to use for some later purpose.
We might exploit that to select the amount of clusters so that it
performs well for that purpose.

For example, if we are clustering t-shirts sizes, to find out the amount
of t-shirt sizes we have to produce, as seen in
\cref{fig:t_shirt_clustering}. We can think about it from the t-shirt
business point of view, and we might conclude that having less t-shirt
sizes might be cheaper, so we would prefer an small amount of clusters.

\stdfig{t_shirt_clustering}{Graphs showing different cluster amounts for the t-shirt industry.}


%% Body end.

%% Bibliography.

    \nocite{*}

\bibliographystyle{plain}
\bibliography{references}

\end{document}